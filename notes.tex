\documentclass[12pt,a4paper]{article}
\usepackage[utf8]{inputenc}
\usepackage[english]{babel}

\usepackage{amsmath, amssymb, amsfonts, physics, braket, mathtools, bigints, geometry}
\usepackage{amsthm}
\usepackage{pgfplots, subcaption, floatrow, footnote, adjustbox,float,fancyvrb, colonequals}
\usepackage{graphicx, grffile, epsfig, listings}
\usepackage{verbatim, dsfont, accents, todonotes}
\usepackage{textcomp}
\usepackage{pdfpages}

\usepackage[dvipsnames]{xcolor}
\usepackage[toc,page]{appendix}
\usepackage{authblk}
\usepackage[bookmarksnumbered=true]{hyperref}
\usepackage{tikz}
\usetikzlibrary{decorations.pathreplacing, patterns}

\usepackage{capt-of, caption} %for captions in minipages
% \usepackage[capitalise]{cleveref}
% \crefname{equation}{}{}
% \usepackage[textsize=footnotesize,textwidth=2.7cm]{todonotes}
% \usepackage[color]{showkeys}

\numberwithin{equation}{section}
\setcounter{tocdepth}{1}
\renewcommand\Affilfont{\itshape\footnotesize}

\title{Notes}
\author[1,*]{DN}
%\affil[3]{ORCID: \href{https://orcid.org/0009-0000-5567-4529}{0009-0000-5567-4529}, e--mail: \href{mailto:diwakar.naidu@unimi.it}{diwakar.naidu@unimi.it}}
%\affil[*]{Università degli Studi di Milano, Via Cesare Saldini 50, 20133 Milano, Italy}

\addtolength{\textwidth}{2.0cm}
\addtolength{\hoffset}{-1.0cm}
\addtolength{\textheight}{2.4cm}
\addtolength{\voffset}{-1.5cm} 

%%%%%%%%%%%%%%%%%%%%%%%%%%%%%%%%%%%%%%%%%%%%%%%%%%%%%%%%%
\newcommand{\bA}{\boldsymbol{A}}
\newcommand{\bB}{\boldsymbol{B}}
\newcommand{\bC}{\boldsymbol{C}}
\newcommand{\bD}{\boldsymbol{D}}
\newcommand{\bE}{\boldsymbol{E}}
\newcommand{\bF}{\boldsymbol{F}}
\newcommand{\cA}{\mathcal{A}}
\newcommand{\cC}{\mathcal{C}}
\newcommand{\cD}{\mathcal{D}}
\newcommand{\cE}{\mathcal{E}}
\newcommand{\cF}{\mathcal{F}}
\newcommand{\cI}{\mathcal{I}}
\newcommand{\cK}{\mathcal{K}}
\newcommand{\cN}{\mathcal{N}}
\newcommand{\cO}{\mathcal{O}}
\newcommand{\cS}{\mathcal{S}}
\newcommand{\fn}{\mathfrak{n}}
\newcommand{\fC}{\mathfrak{C}}
\newcommand{\fR}{\mathfrak{R}}

\newcommand{\CCC}{\mathbb{C}}
\newcommand{\NNN}{\mathbb{N}}
\newcommand{\RRR}{\mathbb{R}}
\newcommand{\TTT}{\mathbb{T}}
\newcommand{\ZZZ}{\mathbb{Z}}
\newcommand{\Zbb}{\mathbb{Z}}

\newcommand{\ulambda}{\underline{\lambda}}

\newcommand{\1}{\mathbb{I}}
\renewcommand{\b}{\textnormal{b}}
\newcommand{\Bog}{\textnormal{Bog}}
\newcommand{\corr}{\textnormal{corr}}
\newcommand{\Coul}{\textnormal{Coul}}
\renewcommand{\d}{\textnormal{d}}
\newcommand{\di}{\textnormal{d}}
\newcommand{\DV}{\mathrm{DV}}
\newcommand{\diam}{\mathrm{diam}}
\newcommand{\eff}{\mathrm{eff}}
\newcommand{\ex}{\mathrm{ex}}
\newcommand{\F}{\mathrm{F}}
\newcommand{\FS}{\mathrm{FS}}
\newcommand{\GS}{\mathrm{gs}}
\newcommand{\HF}{\mathrm{HF}}
\newcommand{\HS}{\mathrm{HS}}
\newcommand{\I}{\mathrm{I}}
\newcommand{\II}{\mathrm{II}}
\newcommand{\III}{\mathrm{III}}
\newcommand{\IV}{\mathrm{IV}}
\newcommand{\V}{\mathrm{V}}
\renewcommand{\Im}{\mathrm{Im}}
\newcommand{\nor}{\mathrm{nor}}
\renewcommand{\Re}{\mathrm{Re}}
\newcommand{\RPA}{\mathrm{RPA}}
\newcommand{\SR}{\mathrm{SR}}
\newcommand{\supp}{\mathrm{supp}}
\newcommand{\trial}{\mathrm{trial}}
\newcommand{\kF}{k_\F}
\newcommand{\BF}{B_\F}
\newcommand{\BFc}{B_\F^c}
\newcommand{\Ik}{\mathcal{I}_k}
\newcommand{\tagg}[1]{ \stepcounter{equation} \tag{\theequation}
	\label{#1} } % add tag and label in align*-environments

\newcommand{\Zstar}{\mathbb{Z}^3} %Make this \Z^3 \setminus \{0\} if needed.

\newcommand{\Rbb}{\mathbb{R}}

\newcommand{\R}{\mathbb{R}}
\newcommand{\C}{\mathbb{C}}
\newcommand{\N}{\mathbb{N}}
\newcommand{\Z}{\mathbb{Z}}
\newcommand{\T}{\mathbb{T}}

\newcommand{\QQ}{\mathcal{Q}}
\newcommand{\h}{\mathfrak{h}}
\newcommand{\HH}{\mathcal{H}}
\newcommand{\FF}{\mathcal{F}}
\newcommand{\LL}{\mathcal{L}}
\newcommand{\KK}{\mathcal{K}}
\newcommand{\NN}{\mathcal{N}}

\newcommand{\SH}{\mathscr{H}}
\newcommand{\Psis}{\Psi^*}
\newcommand{\bint}{\bigintssss}
\newcommand\Item[1][]{%
	\ifx\relax#1\relax  \item \else \item[#1] \fi
	\abovedisplayskip=0pt\abovedisplayshortskip=0pt~\vspace*{-\baselineskip}}
\newcommand{\ep}{\varepsilon}
\newcommand{\dg}{^\dagger}
\newcommand{\half}{\frac{1}{2}}
\newcommand{\eva}[1]{\left\langle #1 \right\rangle}
\newcommand{\bracket}[2]{\left\langle #1 | #2 \right\rangle}
\renewcommand{\det}[1]{\mathrm{det}\left( #1 \right)}
\newcommand{\del}[1]{\frac{\partial}{\partial #1}}
\newcommand{\fulld}[1]{\frac{d}{d #1}}
\newcommand{\fulldd}[2]{\frac{d #1}{d #2}}
\newcommand{\dell}[2]{\frac{\partial #1}{\partial #2}}
\newcommand{\delltwo}[2]{\frac{\partial^2 #1}{\partial #2 ^2}}  
\newcommand{\com}[1]{\left[ #1 \right]}
\newcommand{\floor}[1]{\left\lfloor #1 \right\rfloor}
\newcommand{\normmax}[1]{\norm{#1}_{\max}}
\newcommand{\normmaxi}[1]{\norm{#1}_{\mathrm{max,1}}}
\newcommand{\normmaxii}[1]{\norm{#1}_{\mathrm{max,2}}}
%%%%%%%%%%%%%%%%%%%%%%%%%%%%%%%%%%%%%%%%%%%%%%%%%%%
% THEOREMSTYLES
\theoremstyle{plain}
\newtheorem{theorem}{Theorem}[section]
\newtheorem{lemma}[theorem]{Lemma}
\newtheorem{corollary}[theorem]{Corollary}
\newtheorem{observation}[theorem]{Observation}
\newtheorem{proposition}[theorem]{Proposition}

\theoremstyle{definition}
\newtheorem{definition}[theorem]{Definition}
\newtheorem{problem}[theorem]{Problem}
\newtheorem{assumption}[theorem]{Assumption}
\newtheorem{hypothesis}[theorem]{Hypothesis}
\newtheorem{example}[theorem]{Example}
\newtheorem*{remarks}{Remarks}

\theoremstyle{remark}
\newtheorem{claim}[theorem]{Claim}
\newtheorem{remark}[theorem]{Remark}

% UNNUMBERED VERSIONS
\theoremstyle{plain}
\newtheorem*{theorem*}{Theorem}
\newtheorem*{lemma*}{Lemma}
\newtheorem*{corollary*}{Corollary}
\newtheorem*{proposition*}{Proposition}


\theoremstyle{definition}
\newtheorem*{definition*}{Definition}
\newtheorem*{problem*}{Problem}
\newtheorem*{assumption*}{Assumption}
\newtheorem*{example*}{Example}

\theoremstyle{remark}
\newtheorem*{claim*}{Claim}
\newtheorem*{remark*}{Remark}




\begin{document}
	\maketitle
	\medskip
---------- Here comes a Pedagogical review----------		

This is an introduction to the relevant concepts from the quantum mechanical theory of many body systems.
\section{Introduction}	
Consider a system of $N$ particles described on Hilbert spaces $\h_1,\ldots,\h_N$ with Hamilton operators $h_1,\ldots,h_N$. The combined system of these particles is decribed on the Hilbert space that results from the tensor product of these 1-particle Hilbert spaces, that is 
\begin{equation}
	\HH_N = \h_1\otimes\cdots\otimes\h_N\,.
\end{equation}
Each of the Hamilton operators $h_1,\ldots,h_N$ can be extended to the tensor product spaces as
\begin{equation}
	h_1 \otimes\1\otimes\cdots\otimes\1,\quad \1\otimes h_2\otimes\cdots\otimes\1,\quad \1\otimes\cdots\otimes \1 \otimes h_N\,.
\end{equation}
In the case of free or non-interacting particles, the combined Hamilton operator for the system is $H_N^0 = h_1+h_2+\ldots+h_N$ defined on the domain
\begin{equation}
	D(H_N^0) \coloneq \mathrm{span}\{\phi_1\otimes\ldots\otimes\phi_N \;|\; \phi_1 \in D(h_1), \dots, \phi_N \in D(h_N)\}\,.
\end{equation}
On $\HH_N$, we have a natural action of the symmetric group $S_N$. That is for $\sigma \in S_N$, we have a unitary map $U_\sigma: \HH_N \rightarrow\HH_N $ uniquely defined by its action on the tensor product. So for $\phi = \phi_1\otimes\ldots\otimes\phi_N \in \HH_N$, the $N$-particle wavefunction which is the tensor product of 1-particle wavefunctions, we have
\begin{equation}
	U_\sigma \phi_1\otimes\ldots\otimes\phi_N = \phi_{\sigma^{-1}(1)} \otimes \ldots \otimes \phi_{\sigma^{-1}(N)}\,.
\end{equation}
This unitary map gives rise to two orthogonal projections
\begin{equation}
	P_+ = (N!)^{-1} \sum\limits_{\sigma \in S_N} U_\sigma , \quad  P_- = (N!)^{-1} \sum\limits_{\sigma \in S_N} \mathrm{sgn}(\sigma)U_\sigma\,.
\end{equation}
Then the two projections $P_{\pm}$ define subspaces of $\HH_N$, i.e., the symmetric and antisymmetric tensor products
\begin{equation}
	\bigotimes\limits_{\substack{i=1\\ \mathrm{symm}}}^N\h_i \coloneq P_+(\h_1\otimes\cdots\otimes\h_N) , \quad 
	\bigwedge\limits_{i=1}^N\h_i \coloneq P_-(\h_1\otimes\cdots\otimes\h_N) \,. \label{eq:orthoproj}
\end{equation}
We also define the antisymmetric tensor product of vectors $\phi_1,\ldots,\phi_N \in \h$ as
\begin{equation}
	\phi_1\wedge\ldots\wedge\phi_N \coloneq (N!)^{\half}P_-(\phi_1\otimes\cdots\otimes \phi_N)\,.
\end{equation}
Without delving into the statistics, the distinguishing feature of the two species of particles, bosons and fermions, is that the $N$-particle wavefunction remains unaffected under particle permutations or picks up the sign of the permutation, respectively. Simply, for bosons, we have
\begin{equation}
	\phi(x_1,\ldots,x_i,\ldots,x_j,\ldots,x_N) =\phi(x_1,\ldots,x_j,\ldots,x_i,\ldots,x_N), \; \mathrm{for}\; \phi \in \HH_N, 
\end{equation}
and for fermions, we have
\begin{equation}
	\phi(x_1,\ldots,x_i,\ldots,x_j,\ldots,x_N) =-\phi(x_1,\ldots,x_j,\ldots,x_i,\ldots,x_N), \; \text{for} \;\phi \in \HH_N \,.
\end{equation}


In the physical world, particle systems interact, we denote pair interaction between the particles by $V= V_{i,j}$ with all distinct combinations of $i,j \in \{1,\ldots,N\}$.
In general, we want to study the $N$-body Hamilton operator
\begin{equation}
	H_N \coloneq H_N^0 + \sum\limits_{1\leq i<j\leq N}V_{i,j} 
	%= \sum\limits_{i=1}^{N} \Delta_i 
	\,.
 \end{equation}
acting on N-particle wave function $\phi \in \HH_N$.

Observe that in the definition of the $N$-particle Hilbert space, the particle number stays fixed. If we have an infinte sequence of particles, we can also consider all particle numbers all at once. To this end, we introduce the Fock Hilbert space
\begin{equation}
	\FF = \bigoplus_{N=0}^{\infty} \h_1\otimes\cdots\otimes\h_N
\end{equation}  
where for $N=0$, we interpret $\h_1\otimes\cdots\otimes\h_N$ just as $\C$, being the 0-particle space. The vector $1 \in \C$ is called the vacuum vector and is denoted by $\Omega$ or $\ket{\Omega}$. The Hamilton operator on the Fock space can be written as
\begin{equation}
 H=\bigoplus\limits_{N=0}^{\infty}H_N,\quad H \bigoplus_{N=0}^{\infty} \psi_N = \bigoplus_{N=0}^{\infty} H_N\psi_N, \quad \text{with}\quad H_0 = 0\,,
\end{equation}
with the domain
\begin{equation}
	D\left(H\right) = \left\{\psi = \bigoplus_{N=0}^{\infty} \psi_N\;|\; \psi_N \in D(H_N), \sum\limits_{N=0}^{\infty} \norm{H_N\psi_N}^2<\infty\right\}\,.
\end{equation}
The situation when particle number is fixed is referred as the canonical picture and when we consider all the possible particle number at once is reffered to as grand canonical picture.
%Mentioning stability
In case of identical particles, using the orthogonal projections $P_{\pm}$ from \eqref{eq:orthoproj}, we can define the bosonic and fermionic Fock space as
\begin{align}
	\FF^{\mathrm{B}}(\h) &\coloneq P_{+}(\FF) = \bigoplus_{N=0}^{\infty}\bigotimes^{N}_{sym} \h \,,\\
		\FF^{\mathrm{F}}(\h) &\coloneq P_{-}(\FF) = \bigoplus_{N=0}^{\infty}\bigwedge^{N} \h \,,
\end{align}
with $\h$ being the 1-particle Hilbert space.

\todo{Introduce the exact Hamiltonian we use with the coupling parameter}
Here we choose the 1-particle Hilbert spaces to be $L^2(\R^3)$, which gives us
\begin{equation}
\HH_N = \bigotimes\limits^{N}L^2(\R^3) = L^2(\R^3)^{\otimes N} \simeq L^2(\R^{3N})\,.
\end{equation} 
\subsection{Second quantization}
	\subsubsection{CCR and CCR}
	\subsubsection{2nd Quantisation of 1 and 2-body operators}
	\subsubsection{1 and 2-pdm}
\subsection{Ground state of bosonic and fermionic systems}	
\subsection{Generalized 1-pdm}
\subsection{Bogoliubov transformation}
\subsubsection{Implementability}\
\subsubsection{Diagonalizing gen 1-pdm}
\subsection{Quasi-free states and Quadratic Hamiltonians}
\subsection{Generalized Hartree-Fock and Bogoliubov approximation}
\subsection{Validity of Bogoliubov and Excitation spectrum of a Bose gas}
\subsection{Going beyond Hartree-Fock and Bosonization}
\subsection{Results from BNPSS, BPSS and CHN}
\subsection{Motivation for Momentum distribution and BL}
\hrule
\subsection{Effective equations}
\subsection{About dynamics from BPS}

\end{document}